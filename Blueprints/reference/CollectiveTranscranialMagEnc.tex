\documentclass{article}
\usepackage[utf8]{inputenc}
\usepackage{amsmath}
\usepackage{graphicx}

\title{Optimization of Collective Consciousness Stimulation via Deep Active Inference and Multi-Agent Soft Actor-Critic}
\author{[Ajith Senthil]}
\date{[November 8, 2023]}

\begin{document}

\maketitle

\section*{I. INTRODUCTION}
The landscape of cognitive neuroscience is rapidly evolving with advances in neurostimulation technologies and computational models. Our project represents a pioneering effort to utilize Deep Active Inference (DAI) and Multi-Agent Soft Actor-Critic (mSAC) algorithms to manipulate and understand brain states collectively, propelling us closer to creating a shared rulial space—a concept borrowed from Stephen Wolfram's theory of everything—which manifests as a shared reality constructed through collective learning and mutual information.

\section*{II. PROBLEM STATEMENT}
We aim to devise an algorithmic framework that not only detects and stimulates anomalous brain states in individuals but also facilitates a collective consciousness. The algorithm will harness the power of DAI for anomaly detection and employ mSAC for optimizing stimulation parameters, ensuring safety and efficacy. The end goal is to create a collective learning environment where the discovery of new information by one individual enhances the potential for new information encoding in others, fostering a shared cognitive expansion.

\section*{III. LITERATURE REVIEW}
The integration of DAI with neurostimulation is grounded in the principle of minimizing variational free energy as a proxy for the brain's surprise upon encountering new information. Meanwhile, mSAC, an offshoot of reinforcement learning, is adept at managing complex action spaces and promotes exploration through entropy maximization, making it ideal for coordinating multiple stimulation agents.

\section*{IV. TECHNICAL APPROACH}
\subsection*{1. EEG Data Preprocessing and Artifact Removal:}
Advanced algorithms extract clean EEG signals, serving as inputs for detecting brain state anomalies and guiding stimulation strategies.

\subsection*{2. Simulation and Environment Setup:}
A virtual environment simulates the effects of neurostimulation, with agents representing subsets of stimulation parameters to explore a wide array of brain state inductions.

\subsection*{3. Integration of DAI with mSAC:}
DAI algorithm detects brain state anomalies, while mSAC agents, each corresponding to a stimulation source or an individual's brain, coordinate to optimize the global reward—new information encoding across the collective.

\subsection*{4. Collective Learning Protocol:}
A reward system that not only incentivizes individual learning but also rewards the creation of statistical dependencies among agents, enhancing collective learning.

\subsection*{5. Experiments and Validation:}
Controlled experiments will validate the efficacy of the proposed system in a simulated environment before any real-world application.

\section*{V. INTERMEDIATE/PRELIMINARY RESULTS}
Preliminary simulations have shown the system's capability to detect anomalies and adjust stimulation parameters. Early results indicate that the reward structure based on mutual information can effectively propagate the encoding of new information across multiple agents.

\section*{VI. REFERENCES}
References include key literature on DAI, mSAC, neurostimulation techniques, and collective intelligence paradigms, providing a theoretical foundation for the project.

\end{document}