\documentclass{article}
\usepackage[utf8]{inputenc}
\usepackage{amsmath}
\usepackage{graphicx}

\title{Integrated Anomalous Brain State Detection and Information Encoding via Magnetic and Ultrasound Stimulation using Deep Active Inference and Multi-Agent Soft Actor-Critic}
\author{[Ajith Senthil]}
\date{[November 7, 2023]}

\begin{document}

\maketitle

\section*{I. INTRODUCTION}
Cognitive neuroscience is at the frontier of understanding and influencing brain states. Our project expands upon this premise, applying Deep Active Inference (DAI) and a Multi-Agent Soft Actor-Critic (mSAC) for the detection and induction of anomalous brain states, integrating both transcranial magnetic stimulation (TMS) and transcranial focused ultrasound (tFUS) to encode new information into the brain dynamically.

\section*{II. PROBLEM STATEMENT}
We aim to develop a dual-modality algorithmic framework that not only detects anomalous brain states but also encodes new information through dynamic stimulation. Leveraging the precision of machine learning, this system will adapt stimulation parameters in real time, guided by EEG data feedback, to optimize both safety and effectiveness.

\section*{III. LITERATURE REVIEW}
\subsection*{Deep Active Inference (DAI):}
DAI has been instrumental in optimizing the sampling of information and detecting anomalies. Its application in cognitive neuroscience is anticipated to revolutionize the detection and encoding of new brain states.

\subsection*{Multi-Agent Soft Actor-Critic (mSAC):}
The mSAC algorithm, adept at managing large action spaces and facilitating multi-agent cooperation, will be crucial in dynamically adjusting stimulation parameters across TMS and tFUS modalities.

\subsection*{Transcranial Stimulation Modalities:}
Both TMS and tFUS have individually shown efficacy in modulating brain activity. Their integration with DAI presents an innovative method for inducing and encoding new brain states.

\section*{IV. TECHNICAL APPROACH}
\subsection*{1. EEG Data Preprocessing and Artifact Removal:}
Advanced machine learning techniques will be applied to extract clean EEG signals, crucial for identifying and tracking brain state changes.

\subsection*{2. Simulation and Environment Setup:}
A virtual environment will simulate the effects of TMS and tFUS on brain activity. Agents will be programmed with parameters corresponding to each stimulation modality.

\subsection*{3. Integration of DAI with mSAC:}
DAI will be employed to identify anomalies in EEG-represented brain states, while mSAC will be extended to control the agents governing TMS and tFUS, facilitating the safe and effective manipulation of brain states.

\subsection*{4. Experiments and Validation:}
Controlled experiments will validate the dual-stimulation system's capability in accurate detection and information encoding into brain states. Subject safety and well-being will be closely monitored throughout.

\section*{V. INTERMEDIATE/PRELIMINARY RESULTS}
Initial simulations with the DAI algorithm have successfully detected nuanced brain state alterations. Early integration with the mSAC framework has effectively modulated TMS and tFUS parameters. Ongoing experiments aim to fine-tune the system's safety and functional capacity.

\section*{REFERENCES}
\begin{enumerate}
    \item \textit{Deep Active Inference}: G. Joseph et al., "Anomaly Detection via Deep Active Inference," arXiv preprint arXiv:2105.06288, 2021.
    \item \textit{Soft Actor-Critic}: Haarnoja, T. et al., "Soft Actor-Critic Algorithms and Applications," arXiv preprint arXiv:1812.05905, 2019.
    \item \textit{Transcranial Magnetic Stimulation and Transcranial Focused Ultrasound}: Rossi, S. et al., "Safety, ethical considerations, and application guidelines for the use of transcranial magnetic stimulation and focused ultrasound in clinical practice and research," Clinical Neurophysiology, 2009.
\end{enumerate}

\end{document}
